%% Generated by Sphinx.
\def\sphinxdocclass{report}
\documentclass[letterpaper,10pt,english]{sphinxmanual}
\ifdefined\pdfpxdimen
   \let\sphinxpxdimen\pdfpxdimen\else\newdimen\sphinxpxdimen
\fi \sphinxpxdimen=.75bp\relax

\PassOptionsToPackage{warn}{textcomp}
\usepackage[utf8]{inputenc}
\ifdefined\DeclareUnicodeCharacter
% support both utf8 and utf8x syntaxes
  \ifdefined\DeclareUnicodeCharacterAsOptional
    \def\sphinxDUC#1{\DeclareUnicodeCharacter{"#1}}
  \else
    \let\sphinxDUC\DeclareUnicodeCharacter
  \fi
  \sphinxDUC{00A0}{\nobreakspace}
  \sphinxDUC{2500}{\sphinxunichar{2500}}
  \sphinxDUC{2502}{\sphinxunichar{2502}}
  \sphinxDUC{2514}{\sphinxunichar{2514}}
  \sphinxDUC{251C}{\sphinxunichar{251C}}
  \sphinxDUC{2572}{\textbackslash}
\fi
\usepackage{cmap}
\usepackage[T1]{fontenc}
\usepackage{amsmath,amssymb,amstext}
\usepackage{babel}



\usepackage{times}
\expandafter\ifx\csname T@LGR\endcsname\relax
\else
% LGR was declared as font encoding
  \substitutefont{LGR}{\rmdefault}{cmr}
  \substitutefont{LGR}{\sfdefault}{cmss}
  \substitutefont{LGR}{\ttdefault}{cmtt}
\fi
\expandafter\ifx\csname T@X2\endcsname\relax
  \expandafter\ifx\csname T@T2A\endcsname\relax
  \else
  % T2A was declared as font encoding
    \substitutefont{T2A}{\rmdefault}{cmr}
    \substitutefont{T2A}{\sfdefault}{cmss}
    \substitutefont{T2A}{\ttdefault}{cmtt}
  \fi
\else
% X2 was declared as font encoding
  \substitutefont{X2}{\rmdefault}{cmr}
  \substitutefont{X2}{\sfdefault}{cmss}
  \substitutefont{X2}{\ttdefault}{cmtt}
\fi


\usepackage[Bjarne]{fncychap}
\usepackage{sphinx}

\fvset{fontsize=\small}
\usepackage{geometry}

% Include hyperref last.
\usepackage{hyperref}
% Fix anchor placement for figures with captions.
\usepackage{hypcap}% it must be loaded after hyperref.
% Set up styles of URL: it should be placed after hyperref.
\urlstyle{same}
\addto\captionsenglish{\renewcommand{\contentsname}{Contents:}}

\usepackage{sphinxmessages}
\setcounter{tocdepth}{3}
\setcounter{secnumdepth}{3}


\title{PyNeutron}
\date{Jan 04, 2020}
\release{}
\author{Filip Lew}
\newcommand{\sphinxlogo}{\vbox{}}
\renewcommand{\releasename}{}
\makeindex
\begin{document}

\pagestyle{empty}
\sphinxmaketitle
\pagestyle{plain}
\sphinxtableofcontents
\pagestyle{normal}
\phantomsection\label{\detokenize{index::doc}}



\chapter{src}
\label{\detokenize{modules:src}}\label{\detokenize{modules::doc}}

\section{gui module}
\label{\detokenize{gui:module-gui}}\label{\detokenize{gui:gui-module}}\label{\detokenize{gui::doc}}\index{gui (module)@\spxentry{gui}\spxextra{module}}\index{GameWindow (class in gui)@\spxentry{GameWindow}\spxextra{class in gui}}

\begin{fulllineitems}
\phantomsection\label{\detokenize{gui:gui.GameWindow}}\pysiglinewithargsret{\sphinxbfcode{\sphinxupquote{class }}\sphinxcode{\sphinxupquote{gui.}}\sphinxbfcode{\sphinxupquote{GameWindow}}}{\emph{board}, \emph{*args}, \emph{**kwargs}}{}
Bases: \sphinxcode{\sphinxupquote{PyQt5.QtWidgets.QWidget}}

\end{fulllineitems}

\index{GridCell (class in gui)@\spxentry{GridCell}\spxextra{class in gui}}

\begin{fulllineitems}
\phantomsection\label{\detokenize{gui:gui.GridCell}}\pysiglinewithargsret{\sphinxbfcode{\sphinxupquote{class }}\sphinxcode{\sphinxupquote{gui.}}\sphinxbfcode{\sphinxupquote{GridCell}}}{\emph{pos}, \emph{*args}, \emph{**kwargs}}{}
Bases: \sphinxcode{\sphinxupquote{PyQt5.QtWidgets.QWidget}}
\index{paintEvent() (gui.GridCell method)@\spxentry{paintEvent()}\spxextra{gui.GridCell method}}

\begin{fulllineitems}
\phantomsection\label{\detokenize{gui:gui.GridCell.paintEvent}}\pysiglinewithargsret{\sphinxbfcode{\sphinxupquote{paintEvent}}}{\emph{self}, \emph{QPaintEvent}}{}
\end{fulllineitems}


\end{fulllineitems}



\section{main module}
\label{\detokenize{main:module-main}}\label{\detokenize{main:main-module}}\label{\detokenize{main::doc}}\index{main (module)@\spxentry{main}\spxextra{module}}
User Guide


\section{neutron module}
\label{\detokenize{neutron:module-neutron}}\label{\detokenize{neutron:neutron-module}}\label{\detokenize{neutron::doc}}\index{neutron (module)@\spxentry{neutron}\spxextra{module}}\index{Neutron (class in neutron)@\spxentry{Neutron}\spxextra{class in neutron}}

\begin{fulllineitems}
\phantomsection\label{\detokenize{neutron:neutron.Neutron}}\pysiglinewithargsret{\sphinxbfcode{\sphinxupquote{class }}\sphinxcode{\sphinxupquote{neutron.}}\sphinxbfcode{\sphinxupquote{Neutron}}}{\emph{board}, \emph{position}}{}
Bases: {\hyperref[\detokenize{neutron:neutron.Soldier}]{\sphinxcrossref{\sphinxcode{\sphinxupquote{neutron.Soldier}}}}}

A special case of a {\hyperref[\detokenize{neutron:neutron.Soldier}]{\sphinxcrossref{\sphinxcode{\sphinxupquote{Soldier}}}}}.

A \sphinxcode{\sphinxupquote{Neutron}} is different from a \sphinxcode{\sphinxupquote{Soldier}} only by having a unique color
value.
\index{VALUE (neutron.Neutron attribute)@\spxentry{VALUE}\spxextra{neutron.Neutron attribute}}

\begin{fulllineitems}
\phantomsection\label{\detokenize{neutron:neutron.Neutron.VALUE}}\pysigline{\sphinxbfcode{\sphinxupquote{VALUE}}\sphinxbfcode{\sphinxupquote{ = 1}}}
\end{fulllineitems}


\end{fulllineitems}

\index{NeutronBoard (class in neutron)@\spxentry{NeutronBoard}\spxextra{class in neutron}}

\begin{fulllineitems}
\phantomsection\label{\detokenize{neutron:neutron.NeutronBoard}}\pysigline{\sphinxbfcode{\sphinxupquote{class }}\sphinxcode{\sphinxupquote{neutron.}}\sphinxbfcode{\sphinxupquote{NeutronBoard}}}
Bases: \sphinxcode{\sphinxupquote{object}}

The Neutron game board.

It is represented by a 5x5 NumPy array. The purpose of this class is to
manage the array, ensuring it doesn’t get into an invalid state, and to
provide useful functions for the game’s logic.
\index{grid (neutron.NeutronBoard attribute)@\spxentry{grid}\spxextra{neutron.NeutronBoard attribute}}

\begin{fulllineitems}
\phantomsection\label{\detokenize{neutron:neutron.NeutronBoard.grid}}\pysigline{\sphinxbfcode{\sphinxupquote{grid}}}
the array containing raw data of this board.
\begin{quote}\begin{description}
\item[{Type}] \leavevmode
numpy.array

\end{description}\end{quote}

\end{fulllineitems}

\index{white\_soldiers (neutron.NeutronBoard attribute)@\spxentry{white\_soldiers}\spxextra{neutron.NeutronBoard attribute}}

\begin{fulllineitems}
\phantomsection\label{\detokenize{neutron:neutron.NeutronBoard.white_soldiers}}\pysigline{\sphinxbfcode{\sphinxupquote{white\_soldiers}}}
list of Soldier objects representing white soldiers.
\begin{quote}\begin{description}
\item[{Type}] \leavevmode
list

\end{description}\end{quote}

\end{fulllineitems}

\index{black\_soldiers (neutron.NeutronBoard attribute)@\spxentry{black\_soldiers}\spxextra{neutron.NeutronBoard attribute}}

\begin{fulllineitems}
\phantomsection\label{\detokenize{neutron:neutron.NeutronBoard.black_soldiers}}\pysigline{\sphinxbfcode{\sphinxupquote{black\_soldiers}}}
list of Soldier objects representing black soldiers.
\begin{quote}\begin{description}
\item[{Type}] \leavevmode
list

\end{description}\end{quote}

\end{fulllineitems}

\index{furthest\_empty\_spot() (neutron.NeutronBoard method)@\spxentry{furthest\_empty\_spot()}\spxextra{neutron.NeutronBoard method}}

\begin{fulllineitems}
\phantomsection\label{\detokenize{neutron:neutron.NeutronBoard.furthest_empty_spot}}\pysiglinewithargsret{\sphinxbfcode{\sphinxupquote{furthest\_empty\_spot}}}{\emph{pos}, \emph{dir}}{}
Get the furthest empty position one can get by moving in direction
\sphinxcode{\sphinxupquote{dir}} from position \sphinxcode{\sphinxupquote{pos}} without colliding with anything.

Implemented as a while loop checking following conditions:
\begin{itemize}
\item {} 
if the position after a step in the given direction is still in
the board’s bounds,

\item {} 
if the position after the step is empty.

\end{itemize}

While those conditions are met, the step is performed, adding direction
to position.
If, after executing the loop, the resulting position is different from
the starting position, we return it. Else, the move could not be made,
and we return \sphinxcode{\sphinxupquote{None}}.
\begin{quote}\begin{description}
\item[{Parameters}] \leavevmode\begin{itemize}
\item {} 
\sphinxstyleliteralstrong{\sphinxupquote{pos}} ({\hyperref[\detokenize{util:util.Vec}]{\sphinxcrossref{\sphinxstyleliteralemphasis{\sphinxupquote{util.Vec}}}}}) \textendash{} starting position.

\item {} 
\sphinxstyleliteralstrong{\sphinxupquote{dir}} (\sphinxstyleliteralemphasis{\sphinxupquote{str}}\sphinxstyleliteralemphasis{\sphinxupquote{ or }}{\hyperref[\detokenize{util:util.Vec}]{\sphinxcrossref{\sphinxstyleliteralemphasis{\sphinxupquote{util.Vec}}}}}) \textendash{} direction in which to move.

\end{itemize}

\item[{Returns}] \leavevmode
position of the furthest empty spot in the line of sight
of source position, or \sphinxcode{\sphinxupquote{None}} if the movement cannot be made.

\item[{Return type}] \leavevmode
{\hyperref[\detokenize{util:util.Vec}]{\sphinxcrossref{util.Vec}}}

\end{description}\end{quote}

\end{fulllineitems}

\index{get\_soldiers() (neutron.NeutronBoard method)@\spxentry{get\_soldiers()}\spxextra{neutron.NeutronBoard method}}

\begin{fulllineitems}
\phantomsection\label{\detokenize{neutron:neutron.NeutronBoard.get_soldiers}}\pysiglinewithargsret{\sphinxbfcode{\sphinxupquote{get\_soldiers}}}{\emph{color}}{}
Get all soldiers of a given color present on the board.
\begin{quote}\begin{description}
\item[{Parameters}] \leavevmode
\sphinxstyleliteralstrong{\sphinxupquote{color}} (\sphinxstyleliteralemphasis{\sphinxupquote{int}}) \textendash{} color of the soldiers.

\item[{Returns}] \leavevmode
a list of Soldier objects containing all soldiers of a given color.

\item[{Return type}] \leavevmode
list

\item[{Raises}] \leavevmode
\sphinxstyleliteralstrong{\sphinxupquote{ValueError}} \textendash{} if the color given is not a valid soldier color

\end{description}\end{quote}

\end{fulllineitems}

\index{neighbors() (neutron.NeutronBoard method)@\spxentry{neighbors()}\spxextra{neutron.NeutronBoard method}}

\begin{fulllineitems}
\phantomsection\label{\detokenize{neutron:neutron.NeutronBoard.neighbors}}\pysiglinewithargsret{\sphinxbfcode{\sphinxupquote{neighbors}}}{\emph{pos}}{}
Get values of board cells neighboring cell with position \sphinxcode{\sphinxupquote{pos}}.

It iterates over coordinates from x-1 to x+1 and y-1 to y+1, making
sure they are not out of the bounds of the board, and appends values
at those positions to the resulting list. The source position itself
is not included.
\begin{quote}\begin{description}
\item[{Parameters}] \leavevmode
\sphinxstyleliteralstrong{\sphinxupquote{pos}} ({\hyperref[\detokenize{util:util.Vec}]{\sphinxcrossref{\sphinxstyleliteralemphasis{\sphinxupquote{util.Vec}}}}}) \textendash{} position of the cell.

\item[{Returns}] \leavevmode
list of neighboring cells’ values, without the source cell

\item[{Return type}] \leavevmode
list

\end{description}\end{quote}

\end{fulllineitems}


\end{fulllineitems}

\index{NeutronGame (class in neutron)@\spxentry{NeutronGame}\spxextra{class in neutron}}

\begin{fulllineitems}
\phantomsection\label{\detokenize{neutron:neutron.NeutronGame}}\pysiglinewithargsret{\sphinxbfcode{\sphinxupquote{class }}\sphinxcode{\sphinxupquote{neutron.}}\sphinxbfcode{\sphinxupquote{NeutronGame}}}{\emph{first\_player}, \emph{second\_player}}{}
Bases: \sphinxcode{\sphinxupquote{object}}

The main Neutron game class.
\begin{quote}\begin{description}
\item[{Parameters}] \leavevmode\begin{itemize}
\item {} 
\sphinxstyleliteralstrong{\sphinxupquote{first\_player}} ({\hyperref[\detokenize{player:player.Player}]{\sphinxcrossref{\sphinxstyleliteralemphasis{\sphinxupquote{player.Player}}}}}) \textendash{} the player who will start the game

\item {} 
\sphinxstyleliteralstrong{\sphinxupquote{second\_player}} ({\hyperref[\detokenize{player:player.Player}]{\sphinxcrossref{\sphinxstyleliteralemphasis{\sphinxupquote{player.Player}}}}}) \textendash{} the second player

\end{itemize}

\end{description}\end{quote}
\index{check\_won() (neutron.NeutronGame method)@\spxentry{check\_won()}\spxextra{neutron.NeutronGame method}}

\begin{fulllineitems}
\phantomsection\label{\detokenize{neutron:neutron.NeutronGame.check_won}}\pysiglinewithargsret{\sphinxbfcode{\sphinxupquote{check\_won}}}{}{}
Checks if the game was won, updating self.winner variable with the
color of the winning player.
\begin{quote}\begin{description}
\item[{Returns}] \leavevmode
winning player’s color

\item[{Return type}] \leavevmode
int

\end{description}\end{quote}

\end{fulllineitems}

\index{play\_round() (neutron.NeutronGame method)@\spxentry{play\_round()}\spxextra{neutron.NeutronGame method}}

\begin{fulllineitems}
\phantomsection\label{\detokenize{neutron:neutron.NeutronGame.play_round}}\pysiglinewithargsret{\sphinxbfcode{\sphinxupquote{play\_round}}}{}{}
Plays one round, swapping players afterwards.

\end{fulllineitems}

\index{start() (neutron.NeutronGame method)@\spxentry{start()}\spxextra{neutron.NeutronGame method}}

\begin{fulllineitems}
\phantomsection\label{\detokenize{neutron:neutron.NeutronGame.start}}\pysiglinewithargsret{\sphinxbfcode{\sphinxupquote{start}}}{}{}
Starts the game, playing rounds until the game is won by either of
the players.

\end{fulllineitems}


\end{fulllineitems}

\index{Soldier (class in neutron)@\spxentry{Soldier}\spxextra{class in neutron}}

\begin{fulllineitems}
\phantomsection\label{\detokenize{neutron:neutron.Soldier}}\pysiglinewithargsret{\sphinxbfcode{\sphinxupquote{class }}\sphinxcode{\sphinxupquote{neutron.}}\sphinxbfcode{\sphinxupquote{Soldier}}}{\emph{board}, \emph{position}, \emph{color}}{}
Bases: \sphinxcode{\sphinxupquote{object}}

Class representing a soldier on the board.

Its main task is to enforce proper movement rules, to prevent the board
from getting into an invalid state from the point of view of the game’s
rules.
\begin{quote}\begin{description}
\item[{Parameters}] \leavevmode\begin{itemize}
\item {} 
\sphinxstyleliteralstrong{\sphinxupquote{board}} ({\hyperref[\detokenize{neutron:neutron.NeutronBoard}]{\sphinxcrossref{\sphinxstyleliteralemphasis{\sphinxupquote{neutron.NeutronBoard}}}}}) \textendash{} home board of this Soldier.

\item {} 
\sphinxstyleliteralstrong{\sphinxupquote{position}} ({\hyperref[\detokenize{util:util.Vec}]{\sphinxcrossref{\sphinxstyleliteralemphasis{\sphinxupquote{util.Vec}}}}}) \textendash{} position of this Soldier on the board.

\item {} 
\sphinxstyleliteralstrong{\sphinxupquote{color}} (\sphinxstyleliteralemphasis{\sphinxupquote{int}}) \textendash{} color of this Soldier.

\end{itemize}

\end{description}\end{quote}
\index{move() (neutron.Soldier method)@\spxentry{move()}\spxextra{neutron.Soldier method}}

\begin{fulllineitems}
\phantomsection\label{\detokenize{neutron:neutron.Soldier.move}}\pysiglinewithargsret{\sphinxbfcode{\sphinxupquote{move}}}{\emph{direction}}{}
Tries to move this \sphinxcode{\sphinxupquote{Soldier}} in the given direction.

For this method to succeed, the direction given must be present in
\sphinxcode{\sphinxupquote{self.possible\_directions}}.

Works by calling {\hyperref[\detokenize{neutron:neutron.NeutronBoard.furthest_empty_spot}]{\sphinxcrossref{\sphinxcode{\sphinxupquote{NeutronBoard.furthest\_empty\_spot()}}}}}, setting
the position returned by this function to this Soldier’s color,
and the original position to 0.
\begin{quote}\begin{description}
\item[{Parameters}] \leavevmode
\sphinxstyleliteralstrong{\sphinxupquote{direction}} (\sphinxstyleliteralemphasis{\sphinxupquote{str}}) \textendash{} direction in which to move this Soldier.

\item[{Raises}] \leavevmode
\sphinxstyleliteralstrong{\sphinxupquote{ValueError}} \textendash{} if the given direction is not in self.possible\_directions.

\end{description}\end{quote}

\end{fulllineitems}

\index{possible\_directions() (neutron.Soldier property)@\spxentry{possible\_directions()}\spxextra{neutron.Soldier property}}

\begin{fulllineitems}
\phantomsection\label{\detokenize{neutron:neutron.Soldier.possible_directions}}\pysigline{\sphinxbfcode{\sphinxupquote{property }}\sphinxbfcode{\sphinxupquote{possible\_directions}}}
List of directions this Soldier can move

Works by checking for which directions {\hyperref[\detokenize{neutron:neutron.NeutronBoard.furthest_empty_spot}]{\sphinxcrossref{\sphinxcode{\sphinxupquote{NeutronBoard.furthest\_empty\_spot()}}}}}

\end{fulllineitems}

\index{possible\_moves() (neutron.Soldier property)@\spxentry{possible\_moves()}\spxextra{neutron.Soldier property}}

\begin{fulllineitems}
\phantomsection\label{\detokenize{neutron:neutron.Soldier.possible_moves}}\pysigline{\sphinxbfcode{\sphinxupquote{property }}\sphinxbfcode{\sphinxupquote{possible\_moves}}}
List of positions this Soldier can be after one move.

Works by supplying {\hyperref[\detokenize{neutron:neutron.NeutronBoard.furthest_empty_spot}]{\sphinxcrossref{\sphinxcode{\sphinxupquote{NeutronBoard.furthest\_empty\_spot()}}}}} with all
possible directions, then filtering out \sphinxcode{\sphinxupquote{None}} results.

\end{fulllineitems}


\end{fulllineitems}



\section{player module}
\label{\detokenize{player:module-player}}\label{\detokenize{player:player-module}}\label{\detokenize{player::doc}}\index{player (module)@\spxentry{player}\spxextra{module}}\index{HumanPlayer (class in player)@\spxentry{HumanPlayer}\spxextra{class in player}}

\begin{fulllineitems}
\phantomsection\label{\detokenize{player:player.HumanPlayer}}\pysiglinewithargsret{\sphinxbfcode{\sphinxupquote{class }}\sphinxcode{\sphinxupquote{player.}}\sphinxbfcode{\sphinxupquote{HumanPlayer}}}{\emph{color}}{}
Bases: {\hyperref[\detokenize{player:player.Player}]{\sphinxcrossref{\sphinxcode{\sphinxupquote{player.Player}}}}}
\index{move\_neutron() (player.HumanPlayer method)@\spxentry{move\_neutron()}\spxextra{player.HumanPlayer method}}

\begin{fulllineitems}
\phantomsection\label{\detokenize{player:player.HumanPlayer.move_neutron}}\pysiglinewithargsret{\sphinxbfcode{\sphinxupquote{move\_neutron}}}{\emph{board}}{}
Method called by the game when it’s this player’s turn to move the
neutron.
\begin{quote}\begin{description}
\item[{Parameters}] \leavevmode\begin{itemize}
\item {} 
\sphinxstyleliteralstrong{\sphinxupquote{board}} ({\hyperref[\detokenize{neutron:neutron.NeutronBoard}]{\sphinxcrossref{\sphinxstyleliteralemphasis{\sphinxupquote{neutron.NeutronBoard}}}}}) \textendash{} board of the game played by this

\item {} 
\sphinxstyleliteralstrong{\sphinxupquote{player.}} \textendash{} 

\end{itemize}

\end{description}\end{quote}

\end{fulllineitems}

\index{move\_soldier() (player.HumanPlayer method)@\spxentry{move\_soldier()}\spxextra{player.HumanPlayer method}}

\begin{fulllineitems}
\phantomsection\label{\detokenize{player:player.HumanPlayer.move_soldier}}\pysiglinewithargsret{\sphinxbfcode{\sphinxupquote{move\_soldier}}}{\emph{board}}{}
Method called by the game when it’s this player’s turn to move one of
their soldiers.
\begin{quote}\begin{description}
\item[{Parameters}] \leavevmode\begin{itemize}
\item {} 
\sphinxstyleliteralstrong{\sphinxupquote{board}} ({\hyperref[\detokenize{neutron:neutron.NeutronBoard}]{\sphinxcrossref{\sphinxstyleliteralemphasis{\sphinxupquote{neutron.NeutronBoard}}}}}) \textendash{} board of the game played by this

\item {} 
\sphinxstyleliteralstrong{\sphinxupquote{player.}} \textendash{} 

\end{itemize}

\end{description}\end{quote}

\end{fulllineitems}


\end{fulllineitems}

\index{Player (class in player)@\spxentry{Player}\spxextra{class in player}}

\begin{fulllineitems}
\phantomsection\label{\detokenize{player:player.Player}}\pysiglinewithargsret{\sphinxbfcode{\sphinxupquote{class }}\sphinxcode{\sphinxupquote{player.}}\sphinxbfcode{\sphinxupquote{Player}}}{\emph{color}}{}
Bases: \sphinxcode{\sphinxupquote{abc.ABC}}

Abstract base class of all Neutron game players. Defines methods called by
the game to allow players to make decisions about the next move.
\begin{quote}\begin{description}
\item[{Parameters}] \leavevmode
\sphinxstyleliteralstrong{\sphinxupquote{color}} (\sphinxstyleliteralemphasis{\sphinxupquote{int}}) \textendash{} color of this player’s soldiers.

\end{description}\end{quote}
\index{move\_neutron() (player.Player method)@\spxentry{move\_neutron()}\spxextra{player.Player method}}

\begin{fulllineitems}
\phantomsection\label{\detokenize{player:player.Player.move_neutron}}\pysiglinewithargsret{\sphinxbfcode{\sphinxupquote{abstract }}\sphinxbfcode{\sphinxupquote{move\_neutron}}}{\emph{board}}{}
Method called by the game when it’s this player’s turn to move the
neutron.
\begin{quote}\begin{description}
\item[{Parameters}] \leavevmode\begin{itemize}
\item {} 
\sphinxstyleliteralstrong{\sphinxupquote{board}} ({\hyperref[\detokenize{neutron:neutron.NeutronBoard}]{\sphinxcrossref{\sphinxstyleliteralemphasis{\sphinxupquote{neutron.NeutronBoard}}}}}) \textendash{} board of the game played by this

\item {} 
\sphinxstyleliteralstrong{\sphinxupquote{player.}} \textendash{} 

\end{itemize}

\end{description}\end{quote}

\end{fulllineitems}

\index{move\_soldier() (player.Player method)@\spxentry{move\_soldier()}\spxextra{player.Player method}}

\begin{fulllineitems}
\phantomsection\label{\detokenize{player:player.Player.move_soldier}}\pysiglinewithargsret{\sphinxbfcode{\sphinxupquote{abstract }}\sphinxbfcode{\sphinxupquote{move\_soldier}}}{\emph{board}}{}
Method called by the game when it’s this player’s turn to move one of
their soldiers.
\begin{quote}\begin{description}
\item[{Parameters}] \leavevmode\begin{itemize}
\item {} 
\sphinxstyleliteralstrong{\sphinxupquote{board}} ({\hyperref[\detokenize{neutron:neutron.NeutronBoard}]{\sphinxcrossref{\sphinxstyleliteralemphasis{\sphinxupquote{neutron.NeutronBoard}}}}}) \textendash{} board of the game played by this

\item {} 
\sphinxstyleliteralstrong{\sphinxupquote{player.}} \textendash{} 

\end{itemize}

\end{description}\end{quote}

\end{fulllineitems}


\end{fulllineitems}

\index{RandomPlayer (class in player)@\spxentry{RandomPlayer}\spxextra{class in player}}

\begin{fulllineitems}
\phantomsection\label{\detokenize{player:player.RandomPlayer}}\pysiglinewithargsret{\sphinxbfcode{\sphinxupquote{class }}\sphinxcode{\sphinxupquote{player.}}\sphinxbfcode{\sphinxupquote{RandomPlayer}}}{\emph{color}}{}
Bases: {\hyperref[\detokenize{player:player.Player}]{\sphinxcrossref{\sphinxcode{\sphinxupquote{player.Player}}}}}
\index{move\_neutron() (player.RandomPlayer method)@\spxentry{move\_neutron()}\spxextra{player.RandomPlayer method}}

\begin{fulllineitems}
\phantomsection\label{\detokenize{player:player.RandomPlayer.move_neutron}}\pysiglinewithargsret{\sphinxbfcode{\sphinxupquote{move\_neutron}}}{\emph{board}}{}
Method called by the game when it’s this player’s turn to move the
neutron.
\begin{quote}\begin{description}
\item[{Parameters}] \leavevmode\begin{itemize}
\item {} 
\sphinxstyleliteralstrong{\sphinxupquote{board}} ({\hyperref[\detokenize{neutron:neutron.NeutronBoard}]{\sphinxcrossref{\sphinxstyleliteralemphasis{\sphinxupquote{neutron.NeutronBoard}}}}}) \textendash{} board of the game played by this

\item {} 
\sphinxstyleliteralstrong{\sphinxupquote{player.}} \textendash{} 

\end{itemize}

\end{description}\end{quote}

\end{fulllineitems}

\index{move\_soldier() (player.RandomPlayer method)@\spxentry{move\_soldier()}\spxextra{player.RandomPlayer method}}

\begin{fulllineitems}
\phantomsection\label{\detokenize{player:player.RandomPlayer.move_soldier}}\pysiglinewithargsret{\sphinxbfcode{\sphinxupquote{move\_soldier}}}{\emph{board}}{}
Method called by the game when it’s this player’s turn to move one of
their soldiers.
\begin{quote}\begin{description}
\item[{Parameters}] \leavevmode\begin{itemize}
\item {} 
\sphinxstyleliteralstrong{\sphinxupquote{board}} ({\hyperref[\detokenize{neutron:neutron.NeutronBoard}]{\sphinxcrossref{\sphinxstyleliteralemphasis{\sphinxupquote{neutron.NeutronBoard}}}}}) \textendash{} board of the game played by this

\item {} 
\sphinxstyleliteralstrong{\sphinxupquote{player.}} \textendash{} 

\end{itemize}

\end{description}\end{quote}

\end{fulllineitems}


\end{fulllineitems}

\index{StrategyPlayer (class in player)@\spxentry{StrategyPlayer}\spxextra{class in player}}

\begin{fulllineitems}
\phantomsection\label{\detokenize{player:player.StrategyPlayer}}\pysiglinewithargsret{\sphinxbfcode{\sphinxupquote{class }}\sphinxcode{\sphinxupquote{player.}}\sphinxbfcode{\sphinxupquote{StrategyPlayer}}}{\emph{color}}{}
Bases: {\hyperref[\detokenize{player:player.Player}]{\sphinxcrossref{\sphinxcode{\sphinxupquote{player.Player}}}}}

\end{fulllineitems}



\section{util module}
\label{\detokenize{util:module-util}}\label{\detokenize{util:util-module}}\label{\detokenize{util::doc}}\index{util (module)@\spxentry{util}\spxextra{module}}\index{Color (class in util)@\spxentry{Color}\spxextra{class in util}}

\begin{fulllineitems}
\phantomsection\label{\detokenize{util:util.Color}}\pysigline{\sphinxbfcode{\sphinxupquote{class }}\sphinxcode{\sphinxupquote{util.}}\sphinxbfcode{\sphinxupquote{Color}}}
Bases: \sphinxcode{\sphinxupquote{object}}
\index{BLACK (util.Color attribute)@\spxentry{BLACK}\spxextra{util.Color attribute}}

\begin{fulllineitems}
\phantomsection\label{\detokenize{util:util.Color.BLACK}}\pysigline{\sphinxbfcode{\sphinxupquote{BLACK}}\sphinxbfcode{\sphinxupquote{ = 3}}}
\end{fulllineitems}

\index{WHITE (util.Color attribute)@\spxentry{WHITE}\spxextra{util.Color attribute}}

\begin{fulllineitems}
\phantomsection\label{\detokenize{util:util.Color.WHITE}}\pysigline{\sphinxbfcode{\sphinxupquote{WHITE}}\sphinxbfcode{\sphinxupquote{ = 2}}}
\end{fulllineitems}

\index{color\_names (util.Color attribute)@\spxentry{color\_names}\spxextra{util.Color attribute}}

\begin{fulllineitems}
\phantomsection\label{\detokenize{util:util.Color.color_names}}\pysigline{\sphinxbfcode{\sphinxupquote{color\_names}}\sphinxbfcode{\sphinxupquote{ = \{2: 'white', 3: 'black'\}}}}
\end{fulllineitems}


\end{fulllineitems}

\index{Vec (class in util)@\spxentry{Vec}\spxextra{class in util}}

\begin{fulllineitems}
\phantomsection\label{\detokenize{util:util.Vec}}\pysiglinewithargsret{\sphinxbfcode{\sphinxupquote{class }}\sphinxcode{\sphinxupquote{util.}}\sphinxbfcode{\sphinxupquote{Vec}}}{\emph{x}, \emph{y}}{}
Bases: \sphinxcode{\sphinxupquote{object}}

A very simple implementation of a 2D vector, used to facilitate operations
on positions and directions.
\begin{quote}\begin{description}
\item[{Parameters}] \leavevmode\begin{itemize}
\item {} 
\sphinxstyleliteralstrong{\sphinxupquote{x}} (\sphinxstyleliteralemphasis{\sphinxupquote{int}}) \textendash{} vector’s x coordinate.

\item {} 
\sphinxstyleliteralstrong{\sphinxupquote{y}} (\sphinxstyleliteralemphasis{\sphinxupquote{int}}) \textendash{} vector’s y coordinate.

\end{itemize}

\end{description}\end{quote}
\index{fromtuple() (util.Vec static method)@\spxentry{fromtuple()}\spxextra{util.Vec static method}}

\begin{fulllineitems}
\phantomsection\label{\detokenize{util:util.Vec.fromtuple}}\pysiglinewithargsret{\sphinxbfcode{\sphinxupquote{static }}\sphinxbfcode{\sphinxupquote{fromtuple}}}{\emph{tuple\_pos}}{}
Creates a {\hyperref[\detokenize{util:util.Vec}]{\sphinxcrossref{\sphinxcode{\sphinxupquote{Vec}}}}} from tuple (y, x). This order of coordinates
was chosen to be compatible with NumPy’s way of indexing
multidimensional arrays.
\begin{quote}\begin{description}
\item[{Parameters}] \leavevmode
\sphinxstyleliteralstrong{\sphinxupquote{tuple\_pos}} (\sphinxstyleliteralemphasis{\sphinxupquote{tuple}}) \textendash{} an (y, x) tuple representing vector’s coordinates.

\item[{Returns}] \leavevmode
a newly created Vec.

\item[{Return type}] \leavevmode
{\hyperref[\detokenize{util:util.Vec}]{\sphinxcrossref{Vec}}}

\end{description}\end{quote}

\end{fulllineitems}

\index{x (util.Vec attribute)@\spxentry{x}\spxextra{util.Vec attribute}}

\begin{fulllineitems}
\phantomsection\label{\detokenize{util:util.Vec.x}}\pysigline{\sphinxbfcode{\sphinxupquote{x}}}
\end{fulllineitems}

\index{y (util.Vec attribute)@\spxentry{y}\spxextra{util.Vec attribute}}

\begin{fulllineitems}
\phantomsection\label{\detokenize{util:util.Vec.y}}\pysigline{\sphinxbfcode{\sphinxupquote{y}}}
\end{fulllineitems}


\end{fulllineitems}



\renewcommand{\indexname}{Python Module Index}
\begin{sphinxtheindex}
\let\bigletter\sphinxstyleindexlettergroup
\bigletter{g}
\item\relax\sphinxstyleindexentry{gui}\sphinxstyleindexpageref{gui:\detokenize{module-gui}}
\indexspace
\bigletter{m}
\item\relax\sphinxstyleindexentry{main}\sphinxstyleindexpageref{main:\detokenize{module-main}}
\indexspace
\bigletter{n}
\item\relax\sphinxstyleindexentry{neutron}\sphinxstyleindexpageref{neutron:\detokenize{module-neutron}}
\indexspace
\bigletter{p}
\item\relax\sphinxstyleindexentry{player}\sphinxstyleindexpageref{player:\detokenize{module-player}}
\indexspace
\bigletter{u}
\item\relax\sphinxstyleindexentry{util}\sphinxstyleindexpageref{util:\detokenize{module-util}}
\end{sphinxtheindex}

\renewcommand{\indexname}{Index}
\printindex
\end{document}